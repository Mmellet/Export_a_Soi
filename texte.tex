\documentclass[12pt,french,letterpaper]{article}
%déclaration du document 

\usepackage[T1]{fontenc}
%package pour police
\usepackage[utf8]{inputenc}
%package pour police

\usepackage[document]{ragged2e}
\usepackage{graphicx}
\usepackage{fancyhdr}
\usepackage{biblatex}
\usepackage{csquotes}

\usepackage{mathptmx}%choix de la police mathptmx = Times

\usepackage[dvipsnames,svgnames]{xcolor}%chois pour les couleurs


\usepackage{float}
\let\origfigure=\figure
\let\endorigfigure=\endfigure
\renewenvironment{figure}[1][]{%
  \origfigure[H]
}{%
  \endorigfigure
}

\usepackage{epigraph}
\usepackage{fancyhdr}


\providecommand{\subtitle}[1]{}
\subtitle{correspondance}
\author{Alexya    Morin    Université de Montréal }
\providecommand{\cours}[1]{}
\cours{FRA3825 Pratiques de l'édition numérique}
\cours{}
\date{}


\pagestyle{fancy}
\fancyhf{}
\lhead{}
\rhead{}
\cfoot{Page \thepage}

\usepackage{graphicx}%package pour la gestion des images
\graphicspath{ {./media/} }%chemin des figures

\usepackage{graphicx,grffile}
\makeatletter
\def\maxwidth{\ifdim\Gin@nat@width>\linewidth\linewidth\else\Gin@nat@width\fi}
\def\maxheight{\ifdim\Gin@nat@height>\textheight\textheight\else\Gin@nat@height\fi}
\makeatother
% Scale images if necessary, so that they will not overflow the page
% margins by default, and it is still possible to overwrite the defaults
% using explicit options in \includegraphics[width, height, ...]{}
\setkeys{Gin}{width=\maxwidth,height=\maxheight,keepaspectratio}

\usepackage{amssymb,amsmath}
\usepackage{ifxetex,ifluatex}
\usepackage{fixltx2e} % provides \textsubscript
\ifnum 0\ifxetex 1\fi\ifluatex 1\fi=0 % if pdftex
  \usepackage[T1]{fontenc}


\else % if luatex or xelatex

  \usepackage{unicode-math}

  \defaultfontfeatures{Ligatures=TeX,Scale=MatchLowercase}
\fi
% use upquote if available, for straight quotes in verbatim environments
\IfFileExists{upquote.sty}{\usepackage{upquote}}{}
% use microtype if available
\IfFileExists{microtype.sty}{%
\usepackage[]{microtype}
\UseMicrotypeSet[protrusion]{basicmath} % disable protrusion for tt fonts
}{}
\PassOptionsToPackage{hyphens}{url} % url is loaded by hyperref

\usepackage[unicode=true]{hyperref}
\hypersetup{
            pdftitle={Travail final : correspondance},
            pdfauthor={Alexya ,  Morin,  },
            colorlinks=true,
            urlcolor= RoyalBlue, 
	          linkcolor= Blue,
            pdfborder={0 0 0},
            breaklinks=true}
\urlstyle{same}  % don't use monospace font for urls



\IfFileExists{parskip.sty}{%
\usepackage{parskip}
}{% else
\setlength{\parindent}{0pt}
\setlength{\parskip}{6pt plus 2pt minus 1pt}
}
\setlength{\emergencystretch}{3em}  % prevent overfull lines
\providecommand{\tightlist}{%
  \setlength{\itemsep}{0pt}\setlength{\parskip}{0pt}}
\setcounter{secnumdepth}{0}
% Redefines (sub)paragraphs to behave more like sections
\ifx\paragraph\undefined\else
\let\oldparagraph\paragraph
\renewcommand{\paragraph}[1]{\oldparagraph{#1}\mbox{}}
\fi
\ifx\subparagraph\undefined\else
\let\oldsubparagraph\subparagraph
\renewcommand{\subparagraph}[1]{\oldsubparagraph{#1}\mbox{}}
\fi

% set default figure placement to htbp
\makeatletter
\def\fps@figure{htbp}
\makeatother




\begin{document}%début de mon document

\begin{titlepage}%début de ma page de titre
\begin{center}
    \enlargethispage{2cm}
    
\includegraphics[width = 50mm]{logo} %ajout de l'image en taille logo
%si le logo ne fonctionne pas : 
%\scshape{Université de Montréal} %utilisation des small caps 

\vspace*{3cm}
\scshape\Huge Travail final\\
\normalfont\Large correspondance\\
\large \vspace*{3cm}
Alexya,  Morin,  
\\
\normalsize\vspace*{1cm}Travail présenté dans le cadre du cours FRA3825 - \em Pratiques
de l'édition numérique
 \normalfont donné par Margot Mellet 

\vspace*{3cm}
\end{center}

\vspace*{\fill}
\begin{flushright}
\end{flushright}

\begin{center}
\scshape\normalsize\vspace*{1cm} 21-Mars-2023 --      Université de
Montréal 
\\
\end{center}
\end{titlepage}




\newpage 

\normalsize{Dans le cadre du travail final, je souhaite créer un espace
d'écriture dédié à mes correspondances. Je me suis intéressée à cette
pratique pendant le printemps du premier confinement grâce à des vidéos
sur \href{https://www.youtube.com/watch?v=rIzuVqS3W7o}{YouTube} et
\href{https://www.tiktok.com/@ihearttgiraffes/video/7042062179218017582?is_from_webapp=1\&sender_device=pc\&web_id=7179212705482933766}{TikTok}
qui évoquent le côté artistique et relaxant de ce passe-temps. La
majorité de ce contenu présente les bases de la correspondance liées à
une esthétique soignée qui met en valeur leurs collections personnelles
d'articles de papeterie. Ce projet était un moyen pour moi de contrer
l'ennui et d'améliorer mon anglais à l'écrit en échangeant avec des
anglophones, mais je l'ai délaissé dès que je suis retournée à l'école
en présentiel. J'ai reçu au total douze lettres entre mai et octobre
2020 provenant de sept individus différents qui habitent soit au Canada,
aux États-Unis, en Australie ou en Écosse. Donc, cette plateforme
servira à préserver ces textes dans le but de recommencer à pratiquer
cette activité régulièrement afin d'interagir avec de nouvelles
personnes. Il serait également pertinent de les répertorier selon le
lieu ainsi que l'expéditeur et, si ce n'est pas trop ambitieux, d'avoir
une carte du monde avec des points d'intérêts cliquables associés à
l'adresse d'origine des lettres. De plus, je veux que cet espace adopte
une interface volontairement rétro pour souligner l'aspect désuet de la
correspondance, comme le site \href{https://spacehey.com/}{SpaceHey}.
Lancé en novembre 2020, ce réseau social qui reprend les codes de
Myspace aspire à devenir une alternative aux géants du numérique tels
que Facebook, Instagram ou encore Twitter qui collectent les données
personnelles des utilisateurs. Pour finir, j'ai cherché une plateforme
semblable à mon idée de projet pour le travail final. J'ai trouvé un
\href{https://wish-uwerehere.tumblr.com/}{compte Tumblr} qui publiait de
courts textes sur des choses qui manquaient aux gens pendant la pandémie
sous la forme de cartes postales et de notes. Cette initiative m'a donné
envie d'ajouter des images de tout le contenu à l'intérieur des
enveloppes, incluant notamment les cadeaux, les dessins et les photos
sur mon espace au lieu de seulement retranscrire les lettres.

\hypertarget{bibliographie}{%
\subsection{Bibliographie}\label{bibliographie}}}



\end{document}
