\documentclass[12pt,french,letterpaper]{article}
%déclaration du document 

\usepackage[T1]{fontenc}
%package pour police
\usepackage[utf8]{inputenc}
%package pour police

\usepackage[document]{ragged2e}
\usepackage{graphicx}
\usepackage{fancyhdr}
\usepackage{biblatex}
\usepackage{csquotes}

\usepackage{mathptmx}%choix de la police mathptmx = Times

\usepackage[dvipsnames,svgnames]{xcolor}%chois pour les couleurs


\usepackage{float}
\let\origfigure=\figure
\let\endorigfigure=\endfigure
\renewenvironment{figure}[1][]{%
  \origfigure[H]
}{%
  \endorigfigure
}

\usepackage{epigraph}
\usepackage{fancyhdr}


\providecommand{\subtitle}[1]{}
\subtitle{Carnet de critiques}
\author{Esther    Favroux    Université de Montréal }
\providecommand{\cours}[1]{}
\cours{FRA3825 Pratiques de l'édition numérique}
\cours{}
\date{}


\pagestyle{fancy}
\fancyhf{}
\lhead{}
\rhead{}
\cfoot{Page \thepage}

\usepackage{graphicx}%package pour la gestion des images
\graphicspath{ {./media/} }%chemin des figures

\usepackage{graphicx,grffile}
\makeatletter
\def\maxwidth{\ifdim\Gin@nat@width>\linewidth\linewidth\else\Gin@nat@width\fi}
\def\maxheight{\ifdim\Gin@nat@height>\textheight\textheight\else\Gin@nat@height\fi}
\makeatother
% Scale images if necessary, so that they will not overflow the page
% margins by default, and it is still possible to overwrite the defaults
% using explicit options in \includegraphics[width, height, ...]{}
\setkeys{Gin}{width=\maxwidth,height=\maxheight,keepaspectratio}

\usepackage{amssymb,amsmath}
\usepackage{ifxetex,ifluatex}
\usepackage{fixltx2e} % provides \textsubscript
\ifnum 0\ifxetex 1\fi\ifluatex 1\fi=0 % if pdftex
  \usepackage[T1]{fontenc}


\else % if luatex or xelatex

  \usepackage{unicode-math}

  \defaultfontfeatures{Ligatures=TeX,Scale=MatchLowercase}
\fi
% use upquote if available, for straight quotes in verbatim environments
\IfFileExists{upquote.sty}{\usepackage{upquote}}{}
% use microtype if available
\IfFileExists{microtype.sty}{%
\usepackage[]{microtype}
\UseMicrotypeSet[protrusion]{basicmath} % disable protrusion for tt fonts
}{}
\PassOptionsToPackage{hyphens}{url} % url is loaded by hyperref

\usepackage[unicode=true]{hyperref}
\hypersetup{
            pdftitle={Projet d'écriture : Carnet de critiques},
            pdfauthor={Esther ,  Favroux,  },
            colorlinks=true,
            urlcolor= RoyalBlue, 
	          linkcolor= Blue,
            pdfborder={0 0 0},
            breaklinks=true}
\urlstyle{same}  % don't use monospace font for urls



\IfFileExists{parskip.sty}{%
\usepackage{parskip}
}{% else
\setlength{\parindent}{0pt}
\setlength{\parskip}{6pt plus 2pt minus 1pt}
}
\setlength{\emergencystretch}{3em}  % prevent overfull lines
\providecommand{\tightlist}{%
  \setlength{\itemsep}{0pt}\setlength{\parskip}{0pt}}
\setcounter{secnumdepth}{0}
% Redefines (sub)paragraphs to behave more like sections
\ifx\paragraph\undefined\else
\let\oldparagraph\paragraph
\renewcommand{\paragraph}[1]{\oldparagraph{#1}\mbox{}}
\fi
\ifx\subparagraph\undefined\else
\let\oldsubparagraph\subparagraph
\renewcommand{\subparagraph}[1]{\oldsubparagraph{#1}\mbox{}}
\fi

% set default figure placement to htbp
\makeatletter
\def\fps@figure{htbp}
\makeatother




\begin{document}%début de mon document

\begin{titlepage}%début de ma page de titre
\begin{center}
    \enlargethispage{2cm}
    
\includegraphics[width = 50mm]{logo} %ajout de l'image en taille logo
%si le logo ne fonctionne pas : 
%\scshape{Université de Montréal} %utilisation des small caps 

\vspace*{3cm}
\scshape\Huge Projet d'écriture\\
\normalfont\Large Carnet de critiques\\
\large \vspace*{3cm}
Esther,  Favroux,  
\\
\normalsize\vspace*{1cm}Travail présenté dans le cadre du cours FRA3825 - \em Pratiques
de l'édition numérique
 \normalfont donné par Marie Mellet 

\vspace*{3cm}
\end{center}

\vspace*{\fill}
\begin{flushright}
\end{flushright}

\begin{center}
\scshape\normalsize\vspace*{1cm} 21-Mars-2023 --      Université de
Montréal 
\\
\end{center}
\end{titlepage}




\newpage 

\normalsize{Mon projet d'écriture sera axé autour d'un corpus de
critiques d'œuvres cinématographiques et littéraires. Mon besoin
d'écriture s'inscrit surtout autour de la critique culturelle. Au cours
de ce travail, j'aimerais donc ainsi rassembler des critiques d'œuvres
littéraires ou cinématographiques qui m'ont le plus marquée afin
d'analyser les différents liens entre elles. Qu'est-ce qui les
rapproche, qu'est-ce qui les différencie ? Ce carnet sera donc à la fois
personnel et un petit peu éditorial. C'est un bon exercice d'écriture
également pour moi. Je choisirai entre 3 et 5 œuvres, par exemple :
\emph{Les mots entre mes mains} de Guinevere Glasfurd-Brown,
\emph{Illusions perdues} -- le film -- de Xavier Giannoli, \emph{Le Chat
Potté : la dernière quête} de Joel Crawford, \emph{Mommy} de Xavier
Dolan, \emph{Jeu de patience} de Jennifer Armentrout. C'est une liste
non exhaustive et le choix des œuvres pourrait se voir modifié.
J'aimerais donc diviser ce recueil de critiques en plusieurs sections
distinctes, un peu comme des chapitres. Chaque section comprendra une
seule critique, qu'elle soit littéraire ou cinématographique. Une partie
sera consacrée à l'analyse du corpus. Le modèle 1 conviendra pour mon
projet qui me permettra également d'améliorer ma maîtrise de GitHub. Je
souhaitais tester un nouveau moyen d'édition mais je pense que je ne
maîtrise pas encore assez bien GitHub pour passer à un autre niveau et
c'est la raison pour laquelle je souhaite me concentrer sur cette
méthode d'édition.}



\end{document}
