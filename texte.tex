\documentclass[10pt,french,letterpaper]{article}
\usepackage[T1]{fontenc}
\usepackage[utf8]{inputenc}

\usepackage[french]{babel}

%\usepackage{fontspec}
%\setmainfont{Times New Roman}


\usepackage[a4paper]{geometry}
\geometry{top=2cm, bottom=2cm, left=4cm, right=4cm}
\usepackage[document]{ragged2e}
\usepackage{graphicx}
\usepackage{fancyhdr}
\usepackage{biblatex}
\usepackage{csquotes}

\usepackage{mathptmx}%choix de la police mathptmx = Times

\usepackage[singlespacing]{setspace}

\usepackage[dvipsnames,svgnames]{xcolor}


\pagestyle{fancy}
\fancyhf{}
\rhead{}
\lhead{ }
\cfoot{\footnotesize{ Laforge  
 \\
\thepage}}

\usepackage{amssymb,amsmath}
\usepackage{ifxetex,ifluatex}
\usepackage{fixltx2e} % provides \textsubscript

\renewcommand{\headrulewidth}{0pt}
\renewcommand{\footrulewidth}{0.5pt}

\usepackage{xpatch}
\xapptocmd{\footrule}{\color{black}}{}{}
\xpretocmd{\footrule}{\color{TealBlue}}{}{}


% use upquote if available, for straight quotes in verbatim environments
\IfFileExists{upquote.sty}{\usepackage{upquote}}{}

% use microtype if available
\IfFileExists{microtype.sty}{%
\usepackage[]{microtype}
\UseMicrotypeSet[protrusion]{basicmath} % disable protrusion for tt fonts
}{}
\PassOptionsToPackage{hyphens}{url} % url is loaded by hyperref

\usepackage[unicode=true]{hyperref}
\hypersetup{
            pdftitle={Proposition de projet d'écriture : Carnet
d'écriture pour \emph{Mon Sims}},
            pdfauthor={Gabrielle ,  Laforge,  , },
            colorlinks=true,
            urlcolor= TealBlue, 
	          linkcolor= Emerald,
            pdfborder={0 0 0},
            breaklinks=true}
\urlstyle{same}  % don't use monospace font for urls


\makeatletter
\def\maxwidth{\ifdim\Gin@nat@width>\linewidth\linewidth\else\Gin@nat@width\fi}
\def\maxheight{\ifdim\Gin@nat@height>\textheight\textheight\else\Gin@nat@height\fi}
\makeatother
% Scale images if necessary, so that they will not overflow the page
% margins by default, and it is still possible to overwrite the defaults
% using explicit options in \includegraphics[width, height, ...]{}
\setkeys{Gin}{width=\maxwidth,height=\maxheight,keepaspectratio}

\IfFileExists{parskip.sty}{%
\usepackage{parskip}
}{% else
\setlength{\parindent}{0pt}
\setlength{\parskip}{6pt plus 2pt minus 1pt}
}
\setlength{\emergencystretch}{3em}  % prevent overfull lines
\providecommand{\tightlist}{%
  \setlength{\itemsep}{0pt}\setlength{\parskip}{0pt}}
\setcounter{secnumdepth}{0}
% Redefines (sub)paragraphs to behave more like sections
\ifx\paragraph\undefined\else
\let\oldparagraph\paragraph
\renewcommand{\paragraph}[1]{\oldparagraph{#1}\mbox{}}
\fi
\ifx\subparagraph\undefined\else
\let\oldsubparagraph\subparagraph
\renewcommand{\subparagraph}[1]{\oldsubparagraph{#1}\mbox{}}
\fi

% set default figure placement to htbp
\makeatletter
\def\fps@figure{htbp}
\makeatother


\providecommand{\subtitle}[1]{}
\subtitle{Carnet d'écriture pour \emph{Mon Sims}}
\author{Gabrielle  Laforge }

\providecommand{\cours}[1]{}
\cours{FRA3825 Pratiques de l'édition numérique}
\cours{}
\date{}



\begin{document}

\noindent\rlap{
    \large\textbf{Gabrielle}
    }
\hfill\llap{
  \footnotesize{}
  }
\noindent\rlap{
    \large\textbf{Laforge}
    }
\hfill\llap{
  \footnotesize{}
  }
\noindent\rlap{
    \large\textbf{}
    }
\hfill\llap{
  \footnotesize{Université de Montréal}
  }

\vspace*{0.4cm}
\color{black}\raggedright \Large{\textbf{Proposition de projet
d'écriture : Carnet d'écriture pour \emph{Mon Sims}}}

\vspace*{0.4cm}
\footnotesize{\copyright 2023 par Gabrielle  Laforge  --
Travail présenté dans le cadre du cours FRA3825 - \em Pratiques de
l'édition numérique, 
 \normalfont donné par Margot Mellet.}
\color{TealBlue}\hrule
\color{black}\vspace*{0.4cm}

\normalsize\justifying{Dans le cadre du cours FRA3710 -- \emph{Penser --
Écrire}, j'ai un projet long de création littéraire à rédiger. Ce
travail représente la première fois que je me lance dans une écriture
créative qui dépasse le 10 pages. Pour ce projet, j'ai eu envie
d'explorer l'autofiction et j'ai trouvé l'idée de me raconter par le
biais d'un \emph{Sims}; j'ai voulu exploiter la distance que permet
l'expression de soi à la 3e personne par le fait de me dédoubler en
personnage virtuel dont j'ai le contrôle. Ce dédoublement et cette
distance me permet de proposer une réflexion philosophique sur le
libre-arbitre de l'humain dans un contexte où l'impact des technologies
numériques bouleversent en profondeur les significations qui composent
l'humanité. Dans cette optique, le \emph{Manifeste Cyborg} de Donna
Haraway m'est apparu comme tout à fait relié à mes réflexions. Une
chanson de Regina Spektor,
\emph{\href{https://www.youtube.com/watch?v=-g-dlnH3wc8}{Hero}}, m'a
aussi donné beaucoup d'inspiration dans le cadre de mon projet long,
notamment par les deux extraits de paroles suivants~:

\begin{quote}
``I'm the hero of the story, don't need to be saved;
\end{quote}

\begin{quote}
Power to the people, no we don't want it, we want pleasure.''
\end{quote}

Le principal problème que j'ai rencontré en rédigeant mon projet long
concerne la structure qui sous-tend l'histoire de \emph{Mon Sims}. Pour
le moment, j'y ai été par fragments, chacun d'entre eux séparé
simplement par un astérisque. J'ai aussi séparé le total actuel de 21
pages en 4 chapitres, mais je ne suis pas satisfaite du chemin qui porte
mes 4 chapitres. J'ai le sentiment que le tout est pêle-mêle et ne rend
pas justice aux réflexions qui balisent l'histoire de \emph{Mon Sims}.
C'est pourquoi j'aimerais créer un carnet de création qui appuie mon
projet long~: une sorte de \emph{mindmap} qui pourrait m'aider à
structurer le tout, à prendre en notes et à organiser les idées qui
m'habitent, à leur créer une sorte de banque qui pourrait être
orchestrée selon des schémas visuels. Je crois que l'espace donné dans
le cadre du cours FRA3825 -- \emph{Pratiques de l'édition numérique} est
tout à fait approprié pour un tel carnet. Cela me permettrait de réunir
les nombreuses idées qui m'affleurent, de les mettre en rapport les unes
aux autres, ce qui m'apporterait plus de discernement sur ce qui est
essentiel ou non dans mon projet long. Surtout, cela me permettrait de
visualiser la structure de mon texte et d'en faire des modifications
sans directement provoquer un changement dans le contenu de mon projet
long, afin d'optimiser l'ordre dans lequel je veux présenter ce
contenu.}\raggedright


\end{document}
