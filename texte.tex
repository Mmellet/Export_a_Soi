\documentclass[12pt,french,letterpaper]{article}
%déclaration du document 

\usepackage[T1]{fontenc}
%package pour police
\usepackage[utf8]{inputenc}
%package pour police

\usepackage[document]{ragged2e}
\usepackage{graphicx}
\usepackage{fancyhdr}
\usepackage{biblatex}
\usepackage{csquotes}

\usepackage{mathptmx}%choix de la police mathptmx = Times

\usepackage[dvipsnames,svgnames]{xcolor}%chois pour les couleurs


\usepackage{float}
\let\origfigure=\figure
\let\endorigfigure=\endfigure
\renewenvironment{figure}[1][]{%
  \origfigure[H]
}{%
  \endorigfigure
}

\usepackage{epigraph}
\usepackage{fancyhdr}


\providecommand{\subtitle}[1]{}
\subtitle{Renée Vivien et Sappho}
\author{Emma    Pierron    Université de Montréal }
\providecommand{\cours}[1]{}
\cours{FRA3825 Pratiques de l'édition numérique}
\cours{}
\date{}


\pagestyle{fancy}
\fancyhf{}
\lhead{}
\rhead{}
\cfoot{Page \thepage}

\usepackage{graphicx}%package pour la gestion des images
\graphicspath{ {./media/} }%chemin des figures

\usepackage{graphicx,grffile}
\makeatletter
\def\maxwidth{\ifdim\Gin@nat@width>\linewidth\linewidth\else\Gin@nat@width\fi}
\def\maxheight{\ifdim\Gin@nat@height>\textheight\textheight\else\Gin@nat@height\fi}
\makeatother
% Scale images if necessary, so that they will not overflow the page
% margins by default, and it is still possible to overwrite the defaults
% using explicit options in \includegraphics[width, height, ...]{}
\setkeys{Gin}{width=\maxwidth,height=\maxheight,keepaspectratio}

\usepackage{amssymb,amsmath}
\usepackage{ifxetex,ifluatex}
\usepackage{fixltx2e} % provides \textsubscript
\ifnum 0\ifxetex 1\fi\ifluatex 1\fi=0 % if pdftex
  \usepackage[T1]{fontenc}


\else % if luatex or xelatex

  \usepackage{unicode-math}

  \defaultfontfeatures{Ligatures=TeX,Scale=MatchLowercase}
\fi
% use upquote if available, for straight quotes in verbatim environments
\IfFileExists{upquote.sty}{\usepackage{upquote}}{}
% use microtype if available
\IfFileExists{microtype.sty}{%
\usepackage[]{microtype}
\UseMicrotypeSet[protrusion]{basicmath} % disable protrusion for tt fonts
}{}
\PassOptionsToPackage{hyphens}{url} % url is loaded by hyperref

\usepackage[unicode=true]{hyperref}
\hypersetup{
            pdftitle={Littérature sapphique francophone : Renée Vivien
et Sappho},
            pdfauthor={Emma ,  Pierron,  },
            colorlinks=true,
            urlcolor= RoyalBlue, 
	          linkcolor= Blue,
            pdfborder={0 0 0},
            breaklinks=true}
\urlstyle{same}  % don't use monospace font for urls



\IfFileExists{parskip.sty}{%
\usepackage{parskip}
}{% else
\setlength{\parindent}{0pt}
\setlength{\parskip}{6pt plus 2pt minus 1pt}
}
\setlength{\emergencystretch}{3em}  % prevent overfull lines
\providecommand{\tightlist}{%
  \setlength{\itemsep}{0pt}\setlength{\parskip}{0pt}}
\setcounter{secnumdepth}{0}
% Redefines (sub)paragraphs to behave more like sections
\ifx\paragraph\undefined\else
\let\oldparagraph\paragraph
\renewcommand{\paragraph}[1]{\oldparagraph{#1}\mbox{}}
\fi
\ifx\subparagraph\undefined\else
\let\oldsubparagraph\subparagraph
\renewcommand{\subparagraph}[1]{\oldsubparagraph{#1}\mbox{}}
\fi

% set default figure placement to htbp
\makeatletter
\def\fps@figure{htbp}
\makeatother




\begin{document}%début de mon document

\begin{titlepage}%début de ma page de titre
\begin{center}
    \enlargethispage{2cm}
    
\includegraphics[width = 50mm]{logo} %ajout de l'image en taille logo
%si le logo ne fonctionne pas : 
%\scshape{Université de Montréal} %utilisation des small caps 

\vspace*{3cm}
\scshape\Huge Littérature sapphique francophone\\
\normalfont\Large Renée Vivien et Sappho\\
\large \vspace*{3cm}
Emma,  Pierron,  
\\
\normalsize\vspace*{1cm}Travail présenté dans le cadre du cours FRA3825 - \em Pratiques
de l'édition numérique
 \normalfont donné par Margot Mellet 

\vspace*{3cm}
\end{center}

\vspace*{\fill}
\begin{flushright}
\end{flushright}

\begin{center}
\scshape\normalsize\vspace*{1cm} 21-Mars-2023 --      Université de
Montréal 
\\
\end{center}
\end{titlepage}




\newpage 

\normalsize{\hypertarget{description-du-projet}{%
\subsection{\texorpdfstring{\textbf{Description du
projet}}{Description du projet}}\label{description-du-projet}}

Pour mon projet final, je souhaiterais créer un site web proposant une
réédition numérique des poèmes de la poétesse fançaise du XIXème siècle,
\href{https://fr.wikipedia.org/wiki/Ren\%C3\%A9e_Vivien}{Renée Vivien},
en les mettant en lien avec
\href{https://fr.wikisource.org/wiki/Sapho_(Vivien)}{sa traduction de
Sappho}. Par la suite, je souhaiterais le transformer en sorte de
répertoire de la littérature sapphique francophone, mélant réédition de
textes libres de droits et bibliographies d'auteur.ice.s dont les livres
sont à la vente.

J'aimerais me concentrer principalement sur une édition plus interactive
pour les lecteur.ice.s afin de mettre en avant ce qu'une édition
numérique de textes anciens peut apporter à l'oeuvre, comparée à une
édition imprimée plus classique. J'aimerais proposer si possible une
fonction permettant de faire une recherche par mot pour naviguer dans
les poèmes/textes, plutôt qu'uniquement une recherche par nom ou titre.
La.e Lecteur.ice.s pourrait ainsi voir s'afficher tous les textes
comportant ce mot. Je vais probablement devoir me former un peu à un
autre langage à côté des outils utilisés dans le cadre du cours, comme
HTML voir Python, afin de proposer cette interactivité supplémentaire.
J'aurais également souhaité rendre disponible la fonctionnalité
permettant de convertir les textes en PDF imprimable pour le lectorat,
si cela est faisable dans le temps disponible pour mettre en ligne le
site.

Au niveau de l'organisation de mon site, j'aimerais proposer une section
comportant les textes et poèmes des auteur.rice.s et une autre section
regroupant une biobibliographie des auteur.rice.s. Je souhaiterais
également proposer à l'avenir des retranscriptions d'interviews.

Au niveau esthétique, j'aimerais proposer un site en accord avec le
genre poétique de Renée Vivien et de Sappho et des éditions originales
de leurs oeuvres : au niveau des typographies, des couleurs (je voudrais
mettre en avant le violet associé généralement aux personnes
lesbiennes), de la mise en forme du texte\ldots etc.

Le but de mon projet serait donc à la fois éditorial en proposant une
interactivité propre à l'édition numérique, mais également culturel
puisque la littérature lesbienne, même si elle commence à arriver sur le
devant de la scène, reste une littérature marginalisée.

\hypertarget{inspirations-pour-le-projet}{%
\subsection{\texorpdfstring{\textbf{Inspirations pour le
projet}}{Inspirations pour le projet}}\label{inspirations-pour-le-projet}}

Mon inspiration culturelle principale pour ce projet est le livre
d'Aurore Turbiau, Margot Lachkar, Camille Islert, Manon Berthier et
Alexandre Antolin
\href{http://www.lecavalierbleu.com/livre/ecrire-a-lencre-violette/}{\emph{Écrire
à l'encre violette - Littératures lesbiennes en France de 1900 à nos
jours}}, qui propose une reconstitution et une analyse du corpus de la
littérature lesbienne des XXème et XXIème siècles. Ce livre a été mon
inspiration pour ce projet, et je m'appuierais sur leur corpus à
l'avenir pour augmenter le nombre d'oeuvres présentes sur mon site.

Concernant mon inspiration éditoriale, je n'ai pas encore trouvé de site
ressemblant exactement à ce que je souhaiterais mettre en place. La
plupart des anthologies ou recueils de littérature francophone sont
imprimés, et le site gallica de la BnF qui propose une sorte
d'équivalent en ligne, comme ici avec ce
\href{https://gallica.bnf.fr/html/und/litteratures/femmes-de-lettres?mode=desktop}{répertoire
d'oeuvres d'écrivaines}, ne réédite pas numériquement les textes mais
les numérise.

\newpage

\hypertarget{bibliographie}{%
\subsection*{Bibliographie}\label{bibliographie}}
\addcontentsline{toc}{subsection}{Bibliographie}

\hypertarget{refs}{}
\begin{CSLReferences}{1}{0}
\leavevmode\vadjust pre{\hypertarget{ref-blanc_design_2018}{}}%
Blanc, J. (2018). Design Graphique, Code \& Recherche {[}Site{]}.
\emph{Design graphique, code \& recherche}.
\url{https://julie-blanc.fr/}

\leavevmode\vadjust pre{\hypertarget{ref-bon_tiers_1997}{}}%
Bon, F. (1997). Le {Tiers} Livre, Web \& Littérature {[}Site{]}.
\emph{Tiers Livre}. \url{http://www.tierslivre.net/}

\leavevmode\vadjust pre{\hypertarget{ref-casili_antonio_2009}{}}%
Casili, A. (2009). Antonio {A}. {Casilli Site} {[}Site{]}. \emph{Antonio
A. Casilli Site}. \url{https://www.casilli.fr/}

\leavevmode\vadjust pre{\hypertarget{ref-fauchie_quaternumnet_2019}{}}%
Fauchié, A. (2019). {Quaternum.net} {[}\{Carnet\}{]}.
\emph{quaternum.net}. \url{https://www.quaternum.net}

\leavevmode\vadjust pre{\hypertarget{ref-mellet_blankblue_2020}{}}%
Mellet, M. (2020). Blank.Blue {[}Carnet{]}. \emph{Blank.blue}.
\url{https://blank.blue/}

\leavevmode\vadjust pre{\hypertarget{ref-perret_site_2020}{}}%
Perret, A. (2020). {Site d'Arthur Perret} {[}\{Carnet\}{]}.
\emph{arthurperret.fr}. \url{https://www.arthurperret.fr}

\leavevmode\vadjust pre{\hypertarget{ref-savelli_fenetres_2007}{}}%
Savelli, A. (2007). Fenêtres {Open Space} {[}Carnet{]}. \emph{Fenêtres
Open Space}. \url{https://annesavelli.fr/}

\leavevmode\vadjust pre{\hypertarget{ref-vitali-rosati_culture_2018}{}}%
Vitali-Rosati, M. (2018). Culture Numérique. {Pour} Une Philosophie Du
Numérique {[}Blogue{]}. \emph{Culture numérique.}
\url{http://blog.sens-public.org/marcellovitalirosati/}

\leavevmode\vadjust pre{\hypertarget{ref-walsh_melanie_2016}{}}%
Walsh, M. (2016). Melanie {Walsh} {[}Site{]}. \emph{Melanie Walsh}.
\url{https://melaniewalsh.org/melaniewalsh.org/}

\end{CSLReferences}}



\end{document}
