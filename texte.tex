\documentclass[12pt,french,letterpaper]{article}
%déclaration du document 

\usepackage[T1]{fontenc}
%package pour police
\usepackage[utf8]{inputenc}
%package pour police

\usepackage[document]{ragged2e}
\usepackage{graphicx}
\usepackage{fancyhdr}
\usepackage{biblatex}
\usepackage{csquotes}

\usepackage{mathptmx}%choix de la police mathptmx = Times

\usepackage[dvipsnames,svgnames]{xcolor}%chois pour les couleurs


\usepackage{float}
\let\origfigure=\figure
\let\endorigfigure=\endfigure
\renewenvironment{figure}[1][]{%
  \origfigure[H]
}{%
  \endorigfigure
}

\usepackage{epigraph}
\usepackage{fancyhdr}


\author{Camille    Germain    Université de Montréal }
\providecommand{\cours}[1]{}
\cours{FRA3825 Pratiques de l'édition numérique}
\cours{}
\date{}


\pagestyle{fancy}
\fancyhf{}
\lhead{}
\rhead{}
\cfoot{Page \thepage}

\usepackage{graphicx}%package pour la gestion des images
\graphicspath{ {./media/} }%chemin des figures

\usepackage{graphicx,grffile}
\makeatletter
\def\maxwidth{\ifdim\Gin@nat@width>\linewidth\linewidth\else\Gin@nat@width\fi}
\def\maxheight{\ifdim\Gin@nat@height>\textheight\textheight\else\Gin@nat@height\fi}
\makeatother
% Scale images if necessary, so that they will not overflow the page
% margins by default, and it is still possible to overwrite the defaults
% using explicit options in \includegraphics[width, height, ...]{}
\setkeys{Gin}{width=\maxwidth,height=\maxheight,keepaspectratio}

\usepackage{amssymb,amsmath}
\usepackage{ifxetex,ifluatex}
\usepackage{fixltx2e} % provides \textsubscript
\ifnum 0\ifxetex 1\fi\ifluatex 1\fi=0 % if pdftex
  \usepackage[T1]{fontenc}


\else % if luatex or xelatex

  \usepackage{unicode-math}

  \defaultfontfeatures{Ligatures=TeX,Scale=MatchLowercase}
\fi
% use upquote if available, for straight quotes in verbatim environments
\IfFileExists{upquote.sty}{\usepackage{upquote}}{}
% use microtype if available
\IfFileExists{microtype.sty}{%
\usepackage[]{microtype}
\UseMicrotypeSet[protrusion]{basicmath} % disable protrusion for tt fonts
}{}
\PassOptionsToPackage{hyphens}{url} % url is loaded by hyperref

\usepackage[unicode=true]{hyperref}
\hypersetup{
            pdftitle={Mon carnet d'écriture},
            pdfauthor={Camille ,  Germain,  },
            colorlinks=true,
            urlcolor= RoyalBlue, 
	          linkcolor= Blue,
            pdfborder={0 0 0},
            breaklinks=true}
\urlstyle{same}  % don't use monospace font for urls



\IfFileExists{parskip.sty}{%
\usepackage{parskip}
}{% else
\setlength{\parindent}{0pt}
\setlength{\parskip}{6pt plus 2pt minus 1pt}
}
\setlength{\emergencystretch}{3em}  % prevent overfull lines
\providecommand{\tightlist}{%
  \setlength{\itemsep}{0pt}\setlength{\parskip}{0pt}}
\setcounter{secnumdepth}{0}
% Redefines (sub)paragraphs to behave more like sections
\ifx\paragraph\undefined\else
\let\oldparagraph\paragraph
\renewcommand{\paragraph}[1]{\oldparagraph{#1}\mbox{}}
\fi
\ifx\subparagraph\undefined\else
\let\oldsubparagraph\subparagraph
\renewcommand{\subparagraph}[1]{\oldsubparagraph{#1}\mbox{}}
\fi

% set default figure placement to htbp
\makeatletter
\def\fps@figure{htbp}
\makeatother




\begin{document}%début de mon document

\begin{titlepage}%début de ma page de titre
\begin{center}
    \enlargethispage{2cm}
    
\includegraphics[width = 50mm]{logo} %ajout de l'image en taille logo
%si le logo ne fonctionne pas : 
%\scshape{Université de Montréal} %utilisation des small caps 

\vspace*{3cm}
\scshape\Huge Mon carnet d'écriture\\
\normalfont\Large \\
\large \vspace*{3cm}
Camille,  Germain,  
\\
\normalsize\vspace*{1cm}Travail présenté dans le cadre du cours FRA3825 - \em Pratiques
de l'édition numérique
 \normalfont donné par Margot Mellet 

\vspace*{3cm}
\end{center}

\vspace*{\fill}
\begin{flushright}
\end{flushright}

\begin{center}
\scshape\normalsize\vspace*{1cm} 21-Mars-2023 --      Université de
Montréal 
\\
\end{center}
\end{titlepage}




\newpage 

\normalsize{Mon carnet d'écriture sera construit dans le but d'avancer
mon idée de projet pour la maitrise. J'ai déjà commencé un carnet papier
dans le but de prendre des notes, donc je le continuerais en ligne. Je
compte créer des portraits de femmes, tirées de deux livres : \emph{Les
falaises} de Virginie DeChamplain et \emph{La femme qui fuit} d'Anaïs
Barbeau-Lavalette. Ce seraient des portraits des personnages féminins,
leurs liens, leurs traits leurs caractéristiques et leur histoire. Dans
\emph{Les falaises}, il y a 6 personnages féminins importants et dans
\emph{La femme qui fuit}, il y en a quatre. Ce serait, pour le moment,
surtout concentré sur \emph{Les falaises}, puisque je viens de le lire.
Il pourrait y avoir un espace pour les personnages masculins, mais ce
n'est pas l'élément le plus important pour le moment. Je pourrais aussi
faire un espace pour une courte présentation de chaque autrice avec des
photos.

J'aimerais aussi faire un espace thématique pour certains éléments
présents dans les histoires. Par exemple, dans \emph{Les falaises} l'eau
est un élément important et chaque femme y est comparée d'une façon
différente. J'aborderais entre autres la filiation, la mémoire
collective, le deuil et l'importance du voyage. Il y aurait aussi un
espace pour faire des commentaires de lecture, un peu comme ce qu'a fait
Marcello Vitali-Rosati pour \emph{Phèdre} dans son
\href{http://blog.sens-public.org/marcellovitalirosati/categories/scholia/}{blog}.
Cependant, j'aime moins la façon dont les commentaires ont été
organisés, comme une liste, il faudrait que je trouve une autre façon de
le faire. Finalement, il y aurait un espace pour commenter les sources
trouvées. Si c'est une possibilité de lier une bibliothèque Zotero,
comme on avait fait dans Stylo, ce serait encore mieux. Comme ce sont
des autrices contemporaines, il y a sûrement des ressources numériques
disponibles, ce qui pourrait aussi être un ajout et une source
d'inspiration intéressants.

Au niveau du site, je préfère le deuxième modèle proposé, surtout à
cause de la possibilité d'utiliser des tags. Cela faciliterait la
recherche dans les différentes sections du carnet. Dans l'exemple, il y
a une section qui les regroupe, ce qui rend la recherche vraiment
facile, aussi dans l'onglet
«\href{https://cupper-hugo-theme.netlify.app/post/}{Blog}» il y a une
barre de recherche, ce que je trouve utile. Peut-être qu'en ayant
simplement une barre de recherche, les tags deviendraient obsolètes?
C'est à explorer. Mon objectif est de continuer à utiliser et à fournir
le carnet tout au long de mon mémoire, dans un but de collecte de
données, d'organisation et afin de faciliter l'accès à l'information.
J'aimerais donner à chaque section une couleur différentes, ce qui
rendrait le carnet plus dynamique et stimulant. Je voudrais aussi
combler l'espace du logo avec un personnalisé, il y a quelques sites en
ligne qui peuvent en produire.}



\end{document}
