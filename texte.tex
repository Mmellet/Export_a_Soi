\documentclass[12pt,french,letterpaper]{article}
%déclaration du document 

\usepackage[T1]{fontenc}
%package pour police
\usepackage[utf8]{inputenc}
%package pour police

\usepackage[document]{ragged2e}
\usepackage{graphicx}
\usepackage{fancyhdr}
\usepackage{biblatex}
\usepackage{csquotes}

\usepackage{mathptmx}%choix de la police mathptmx = Times

\usepackage[dvipsnames,svgnames]{xcolor}%chois pour les couleurs


\usepackage{float}
\let\origfigure=\figure
\let\endorigfigure=\endfigure
\renewenvironment{figure}[1][]{%
  \origfigure[H]
}{%
  \endorigfigure
}

\usepackage{epigraph}
\usepackage{fancyhdr}


\providecommand{\subtitle}[1]{}
\subtitle{Proposition de projet d'écriture}
\author{Amélie    Levasseur-Raymond    Université de Montréal }
\providecommand{\cours}[1]{}
\cours{FRA3825 Pratiques de l'édition numérique}
\cours{}
\date{}


\pagestyle{fancy}
\fancyhf{}
\lhead{}
\rhead{}
\cfoot{Page \thepage}

\usepackage{graphicx}%package pour la gestion des images
\graphicspath{ {./media/} }%chemin des figures

\usepackage{graphicx,grffile}
\makeatletter
\def\maxwidth{\ifdim\Gin@nat@width>\linewidth\linewidth\else\Gin@nat@width\fi}
\def\maxheight{\ifdim\Gin@nat@height>\textheight\textheight\else\Gin@nat@height\fi}
\makeatother
% Scale images if necessary, so that they will not overflow the page
% margins by default, and it is still possible to overwrite the defaults
% using explicit options in \includegraphics[width, height, ...]{}
\setkeys{Gin}{width=\maxwidth,height=\maxheight,keepaspectratio}

\usepackage{amssymb,amsmath}
\usepackage{ifxetex,ifluatex}
\usepackage{fixltx2e} % provides \textsubscript
\ifnum 0\ifxetex 1\fi\ifluatex 1\fi=0 % if pdftex
  \usepackage[T1]{fontenc}


\else % if luatex or xelatex

  \usepackage{unicode-math}

  \defaultfontfeatures{Ligatures=TeX,Scale=MatchLowercase}
\fi
% use upquote if available, for straight quotes in verbatim environments
\IfFileExists{upquote.sty}{\usepackage{upquote}}{}
% use microtype if available
\IfFileExists{microtype.sty}{%
\usepackage[]{microtype}
\UseMicrotypeSet[protrusion]{basicmath} % disable protrusion for tt fonts
}{}
\PassOptionsToPackage{hyphens}{url} % url is loaded by hyperref

\usepackage[unicode=true]{hyperref}
\hypersetup{
            pdftitle={D'infographiste à archiviste -- Fusion entre le papier et le numérique : Proposition
de projet d'écriture},
            pdfauthor={Amélie   Levasseur-Raymond  },
            colorlinks=true,
            urlcolor= RoyalBlue, 
	          linkcolor= Blue,
            pdfborder={0 0 0},
            breaklinks=true}
\urlstyle{same}  % don't use monospace font for urls



\IfFileExists{parskip.sty}{%
\usepackage{parskip}
}{% else
\setlength{\parindent}{0pt}
\setlength{\parskip}{6pt plus 2pt minus 1pt}
}
\setlength{\emergencystretch}{3em}  % prevent overfull lines
\providecommand{\tightlist}{%
  \setlength{\itemsep}{0pt}\setlength{\parskip}{0pt}}
\setcounter{secnumdepth}{0}
% Redefines (sub)paragraphs to behave more like sections
\ifx\paragraph\undefined\else
\let\oldparagraph\paragraph
\renewcommand{\paragraph}[1]{\oldparagraph{#1}\mbox{}}
\fi
\ifx\subparagraph\undefined\else
\let\oldsubparagraph\subparagraph
\renewcommand{\subparagraph}[1]{\oldsubparagraph{#1}\mbox{}}
\fi

% set default figure placement to htbp
\makeatletter
\def\fps@figure{htbp}
\makeatother



% prevent hyphenation
\hyphenpenalty=10000

\begin{document}%début de mon document

\begin{titlepage}%début de ma page de titre
\begin{center}
    \enlargethispage{2cm}
    
\includegraphics[width = 50mm]{logo} %ajout de l'image en taille logo
%si le logo ne fonctionne pas : 
%\scshape{Université de Montréal} %utilisation des small caps 

\vspace*{3cm}
\scshape\Huge D'infographiste à archiviste -- Fusion entre le papier et
le numérique\\
\normalfont\Large Proposition de projet d'écriture\\
\large \vspace*{3cm}
Amélie Levasseur-Raymond 
\\
\normalsize\vspace*{1cm}Travail présenté dans le cadre du cours FRA3825 - \em Pratiques
de l'édition numérique
 \normalfont donné par Margot Mellet 

\vspace*{3cm}
\end{center}

\vspace*{\fill}
\begin{flushright}
\end{flushright}

\begin{center}
\scshape\normalsize\vspace*{1cm} 21-Mars-2023 --      Université de
Montréal 
\\
\end{center}
\end{titlepage}




\newpage 

\normalsize{\hypertarget{besoin-duxe9criture}{%
\subsection{Besoin d'écriture}\label{besoin-duxe9criture}}

Ayant passé 20 ans de ma carrière comme infographiste, ce sont surtout
les aspects visuel et technique que j'aimerais explorer.
\href{http://amelielr.ca}{Mon espace web actuel} est bâti avec le CMS
Drupal, mais avec le temps, certaines fonctionnalités se sont brisées et
ne me permettent plus d'afficher ou même de gérer mon site comme je
l'avais conçu au départ. Cet espace personnel est donc dû pour une
refonte, autant du point de vue technique afin de le simplifier, que
dans son contenu pour y ajouter les branches issues de mon nouveau
parcours en archivistique et en humanités numériques. Je souhaite y
présenter les projets accomplis et en développement, mais aussi pouvoir
m'en servir comme recueil de publications, d'outils ou de toute autre
référence pertinente, question de regrouper et classer tout ce qui
pourrait être utile à un seul et même endroit, un peu à la manière de
Pinterest.

Bien qu'à long terme je compte pouvoir y présenter ces multiples
facettes et intérêts qui me définissent en me servant de cet espace
comme portfolio et pour le partage d'information, je me concentrerai
principalement, dans le cadre de cet exercice, à bâtir la structure et
définir le style qui me représentera. Pour ce qui est du contenu, je
m'attarderai d'abord à rapatrier l'information publiée sur ma page
Facebook, \href{http://facebook.com/1nfograph3}{L'1nfograph3}, afin d'en
assurer la pérennité. J'espère ensuite construire une page qui me
permettra d'y rassembler diverses références que je pourrai classer par
sujet. J'utiliserai pour se faire un recueil de lois monté dans Zotero
dans le cadre du cours \emph{ARV3053-Aspects juridiques des archives et
de l'information} suivi à la session d'hiver 2022. Le reste de
l'arborescence (présentation, cv, portfolio, projets, etc.) ne sera que
partiel selon le temps que j'aurai à y consacrer.

\hypertarget{objectifs}{%
\subsection{Objectifs}\label{objectifs}}

Mon but sera donc principalement d'expérimenter les divers outils vus
durant la session (et mes autres cours) afin de voir comment structurer
une écriture sans devoir mettre de côté la créativité visuelle. Jusqu'à
présent, je découvre que le langage Markdown est en effet facile
d'utilisation pour sémantiser les textes, mais j'aimerais pousser plus
loin afin de voir comment l'intégrer aux autres outils pour créer des
pages ayant une structure plus libre et personnelle. Considérant la
contrainte de thème, j'espère tout de même réussir à détourner le thème
choisi (Cupper) afin de le rendre à mon image et adapté à mes besoins.
Mes connaissances en html/css me seront utiles ici.

\hypertarget{inspiration}{%
\subsection{Inspiration}\label{inspiration}}

Pour ce qui est de l'inspiration, je partirai bien sûr de l'image que
j'ai déjà développée, mais je souhaite tout de même la mettre à jour en
retravaillant la palette de couleurs. J'aime particulièrement les fonds
plus foncés (pour leur économie d'énergie à l'affichage), comme c'est le
cas dans le carnet d'Antoine Fauchié (Fauchié, 2019). La typographie
restera la même, Roboto, mais je pense peut-être pencher davantage vers
le Roboto Slab qui s'apparente à la police de caractère Courier pour
aller vers un style plus rétro numérique, en souvenir de mes débuts sur
WordPerfect il y a de cela bien longtemps déjà il me semble. Les formes
géographiques, combinées à du code ASCII pour une touche numérique et
une texture de papier pour rappeler les archives pourraient être une
dualité intéressante.

\begin{figure}
\centering
\includegraphics{media/ALR_logo.jpg}
\caption{Mon image et logo actuels}
\end{figure}

\newpage

\hypertarget{bibliographie}{%
\subsection*{Bibliographie}\label{bibliographie}}
\addcontentsline{toc}{subsection}{Bibliographie}

\hypertarget{refs}{}
\begin{CSLReferences}{1}{0}
\leavevmode\vadjust pre{\hypertarget{ref-blanc_design_2018}{}}%
Blanc, J. (2018). Design Graphique, Code \& Recherche {[}Site{]}.
\emph{Design graphique, code \& recherche}.
\url{https://julie-blanc.fr/}

\leavevmode\vadjust pre{\hypertarget{ref-bon_tiers_1997}{}}%
Bon, F. (1997). Le {Tiers} Livre, Web \& Littérature {[}Site{]}.
\emph{Tiers Livre}. \url{http://www.tierslivre.net/}

\leavevmode\vadjust pre{\hypertarget{ref-casili_antonio_2009}{}}%
Casili, A. (2009). Antonio {A}. {Casilli Site} {[}Site{]}. \emph{Antonio
A. Casilli Site}. \url{https://www.casilli.fr/}

\leavevmode\vadjust pre{\hypertarget{ref-fauchie_quaternumnet_2019}{}}%
Fauchié, A. (2019). {Quaternum.net} {[}\{Carnet\}{]}.
\emph{quaternum.net}. \url{https://www.quaternum.net}

\leavevmode\vadjust pre{\hypertarget{ref-mellet_blankblue_2020}{}}%
Mellet, M. (2020). Blank.Blue {[}Carnet{]}. \emph{Blank.blue}.
\url{https://blank.blue/}

\leavevmode\vadjust pre{\hypertarget{ref-perret_site_2020}{}}%
Perret, A. (2020). {Site d'Arthur Perret} {[}\{Carnet\}{]}.
\emph{arthurperret.fr}. \url{https://www.arthurperret.fr}

\leavevmode\vadjust pre{\hypertarget{ref-savelli_fenetres_2007}{}}%
Savelli, A. (2007). Fenêtres {Open Space} {[}Carnet{]}. \emph{Fenêtres
Open Space}. \url{https://annesavelli.fr/}

\leavevmode\vadjust pre{\hypertarget{ref-vitali-rosati_culture_2018}{}}%
Vitali-Rosati, M. (2018). Culture Numérique. {Pour} Une Philosophie Du
Numérique {[}Blogue{]}. \emph{Culture numérique.}
\url{http://blog.sens-public.org/marcellovitalirosati/}

\leavevmode\vadjust pre{\hypertarget{ref-walsh_melanie_2016}{}}%
Walsh, M. (2016). Melanie {Walsh} {[}Site{]}. \emph{Melanie Walsh}.
\url{https://melaniewalsh.org/}

\end{CSLReferences}}



\end{document}
