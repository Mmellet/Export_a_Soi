\documentclass[12pt,french,letterpaper]{article}
%déclaration du document 

\usepackage[T1]{fontenc}
%package pour police
\usepackage[utf8]{inputenc}
%package pour police

\usepackage[document]{ragged2e}
\usepackage{graphicx}
\usepackage{fancyhdr}
\usepackage{biblatex}
\usepackage{csquotes}

\usepackage{mathptmx}%choix de la police mathptmx = Times

\usepackage[dvipsnames,svgnames]{xcolor}%chois pour les couleurs


\usepackage{float}
\let\origfigure=\figure
\let\endorigfigure=\endfigure
\renewenvironment{figure}[1][]{%
  \origfigure[H]
}{%
  \endorigfigure
}

\usepackage{epigraph}
\usepackage{fancyhdr}


\providecommand{\subtitle}[1]{}
\subtitle{un zine cinématographique}
\author{Ludi    Marwood    Université de Montréal }
\providecommand{\cours}[1]{}
\cours{FRA3825 Pratiques de l'édition numérique}
\cours{}
\date{}


\pagestyle{fancy}
\fancyhf{}
\lhead{}
\rhead{}
\cfoot{Page \thepage}

\usepackage{graphicx}%package pour la gestion des images
\graphicspath{ {./media/} }%chemin des figures

\usepackage{graphicx,grffile}
\makeatletter
\def\maxwidth{\ifdim\Gin@nat@width>\linewidth\linewidth\else\Gin@nat@width\fi}
\def\maxheight{\ifdim\Gin@nat@height>\textheight\textheight\else\Gin@nat@height\fi}
\makeatother
% Scale images if necessary, so that they will not overflow the page
% margins by default, and it is still possible to overwrite the defaults
% using explicit options in \includegraphics[width, height, ...]{}
\setkeys{Gin}{width=\maxwidth,height=\maxheight,keepaspectratio}

\usepackage{amssymb,amsmath}
\usepackage{ifxetex,ifluatex}
\usepackage{fixltx2e} % provides \textsubscript
\ifnum 0\ifxetex 1\fi\ifluatex 1\fi=0 % if pdftex
  \usepackage[T1]{fontenc}


\else % if luatex or xelatex

  \usepackage{unicode-math}

  \defaultfontfeatures{Ligatures=TeX,Scale=MatchLowercase}
\fi
% use upquote if available, for straight quotes in verbatim environments
\IfFileExists{upquote.sty}{\usepackage{upquote}}{}
% use microtype if available
\IfFileExists{microtype.sty}{%
\usepackage[]{microtype}
\UseMicrotypeSet[protrusion]{basicmath} % disable protrusion for tt fonts
}{}
\PassOptionsToPackage{hyphens}{url} % url is loaded by hyperref

\usepackage[unicode=true]{hyperref}
\hypersetup{
            pdftitle={Je(s) : un zine cinématographique},
            pdfauthor={Ludi ,  Marwood,  },
            colorlinks=true,
            urlcolor= RoyalBlue, 
	          linkcolor= Blue,
            pdfborder={0 0 0},
            breaklinks=true}
\urlstyle{same}  % don't use monospace font for urls



\IfFileExists{parskip.sty}{%
\usepackage{parskip}
}{% else
\setlength{\parindent}{0pt}
\setlength{\parskip}{6pt plus 2pt minus 1pt}
}
\setlength{\emergencystretch}{3em}  % prevent overfull lines
\providecommand{\tightlist}{%
  \setlength{\itemsep}{0pt}\setlength{\parskip}{0pt}}
\setcounter{secnumdepth}{0}
% Redefines (sub)paragraphs to behave more like sections
\ifx\paragraph\undefined\else
\let\oldparagraph\paragraph
\renewcommand{\paragraph}[1]{\oldparagraph{#1}\mbox{}}
\fi
\ifx\subparagraph\undefined\else
\let\oldsubparagraph\subparagraph
\renewcommand{\subparagraph}[1]{\oldsubparagraph{#1}\mbox{}}
\fi

% set default figure placement to htbp
\makeatletter
\def\fps@figure{htbp}
\makeatother




\begin{document}%début de mon document

\begin{titlepage}%début de ma page de titre
\begin{center}
    \enlargethispage{2cm}
    
\includegraphics[width = 50mm]{logo} %ajout de l'image en taille logo
%si le logo ne fonctionne pas : 
%\scshape{Université de Montréal} %utilisation des small caps 

\vspace*{3cm}
\scshape\Huge Je(s)\\
\normalfont\Large un zine cinématographique\\
\large \vspace*{3cm}
Ludi,  Marwood,  
\\
\normalsize\vspace*{1cm}Travail présenté dans le cadre du cours FRA3825 - \em Pratiques
de l'édition numérique
 \normalfont donné par Margot Mellet 

\vspace*{3cm}
\end{center}

\vspace*{\fill}
\begin{flushright}
\end{flushright}

\begin{center}
\scshape\normalsize\vspace*{1cm} 21-Mars-2023 --      Université de
Montréal 
\\
\end{center}
\end{titlepage}




\newpage 

\normalsize{\hypertarget{point-de-duxe9part}{%
\subsection{Point de départ}\label{point-de-duxe9part}}

Mon projet est motivé par un projet externe à ce cours : la création
d'un zine de critiques cinématographiques avec des ami.e.s. Il me semble
pertinent de le présenter en deux parties: la première abordera le
concept général de notre zine (le fil rouge), la deuxième exposera les
volontés et questionnements concernant la création d'une plateforme
numérique pour ée zine (nos besoins).

\hypertarget{le-projet}{%
\subsection{Le projet}\label{le-projet}}

Nous étudions tous.te.s en cinéma et écrivons des critiques
cinématographique. Nous avons remarqué que la critique actuelle était
très limitée. Tout d'abord par une standardisation de sa forme : résumé
de l'œuvre, analyse de l'objet filmique, conclusion passant par une
petite opinion du film. Puis dans sa prise de position : l'influence des
maisons de production aujourd'hui (entres autres) exerce un pouvoir de «
censure » sur les critiques cinématographiques, les empêchant de donner
réellement un avis sur les films sous peine de se voir refuser l'entrée
dans des salles de cinéma ou à des festivals (comme l'ont vécu Alexandre
Fontaine-Rousseau et Denis Côté). Enfin par une volonté « d'objectivité
» dans les critiques, terme que nous remettons en cause non seulement
car il n'est pas possible de s'absoudre de soi-même mais également parce
que refuser de se positionner est plus « subjectif » qu'objectif puisque
c'est un refus de regarder un objet dans ce qu'il propose à travers de
ressentis « objectifs ». Notre projet est donc de créer un zine qui
travaillera des critiques libérées dans leurs opinions, comme dans leurs
formes qui auront pour volonté de refléter l'objet filmique étudié
(n'est-ce pas Godard qui disait qu'il fallait passer par la poésie pour
parler des films ?)

\hypertarget{le-site-web}{%
\subsection{Le site web}\label{le-site-web}}

Afin de ne pas être soumis.e.s aux « contraintes » des maisons de
production ou de tomber dans une logique de capitalisation de la
culture, nous souhaitons que notre projet reste à petite échelle, nous
ne publierons des zines qu'en petites quantités, feront attention à nos
choix matériels (papiers recyclés,\ldots) et de distribution. Nous nous
sommes demandé quelle serait l'utilité et les enjeux éthiques de créer
une plateforme numérique dans laquelle nous publierions des critiques.
Deux points sont ressortis de ces questionnements : le premier étant
que, si plateforme numérique il y avait, cette dernière ne devrait pas
passer par des plateformes dominantes (Google, etc) afin de ne pas être
soumise à des instances que nous rejetons. Le deuxième point était :
comment accueillir le numérique sans tomber dans la surproduction et la
sur-publication ? Comment créer une plateforme qui nous permettrait de
garder ce rythme restreint et intime que nous voulons appliquer au zine
? Nous voulons mettre en avant chaque critique sans tomber dans une
logique de flux, lui laissant une visibilité, une singularité, ne
hiérarchisant pas les films (par exemple, un classement des films par
leurs dates de sortie/mouvements cinématographiques auxquels ils se
rattachent, aurait comme effets une supériorité cinéphile de certains
films étudiés, ces derniers jouissant du statut de « films cultes »), en
bref nous voulons laisser à chaque critique ainsi qu'à chaque objet
filmique la place d'exister dans sa proposition .

Ces deux questionnements me permettront, pour le premier, de travailler
le format de la plateforme, qui passera par des logiciels gratuits et vu
en cours, et le deuxième la sémantisation de la plateforme. J'aimerais
créer une page d'accueil sur laquelle les critiques ne seront pas «
classées » ou « listées » mais rassemblées en une sorte de patchwork ou
de mosaïque d'affiches de films, classées « aléatoirement » comme cela
nous chantera (par couleur, assemblées pour créer des formes,etc). Ainsi
un.e visiteur.euse du site pourra cliquer sur l'affiche qu'iel souhaite
et se rendre sur la critique du film. Au fil du temps, nous pourrons
rajouter des films à cette mosaïque, déplacer certaines affiches,
trouver de nouvelles formes, etc. Tout en gardant sa singularité,
l'affiche du film participera à la création de quelque chose de plus
grand.

\hypertarget{inspirations}{%
\subsection{Inspirations}\label{inspirations}}

Je m'inspirerai notamment des plateformes web de critiques de
\href{https://www.panorama-cinema.com/V2/index.php}{Panorama Cinéma},
\href{https://revue24images.com}{24images}, la revue française
\href{https://tsounami.fr}{Tsunami} mais surtout la revue
\href{https://horschamp.qc.ca}{Hors champ} qui nous a inspiré dans notre
projet, notamment dans sa volonté d'explorer le cinéma (ou plutôt les
écritures du cinéma) par différentes formes.

\hypertarget{refs}{}
\begin{CSLReferences}{1}{0}
\leavevmode\vadjust pre{\hypertarget{ref-blanc_design_2018}{}}%
Blanc, J. (2018). Design Graphique, Code \& Recherche {[}Site{]}.
\emph{Design graphique, code \& recherche}.
\url{https://julie-blanc.fr/}

\leavevmode\vadjust pre{\hypertarget{ref-bon_tiers_1997}{}}%
Bon, F. (1997). Le {Tiers} Livre, Web \& Littérature {[}Site{]}.
\emph{Tiers Livre}. \url{http://www.tierslivre.net/}

\leavevmode\vadjust pre{\hypertarget{ref-casili_antonio_2009}{}}%
Casili, A. (2009). Antonio {A}. {Casilli Site} {[}Site{]}. \emph{Antonio
A. Casilli Site}. \url{https://www.casilli.fr/}

\leavevmode\vadjust pre{\hypertarget{ref-fauchie_quaternumnet_2019}{}}%
Fauchié, A. (2019). {Quaternum.net} {[}\{Carnet\}{]}.
\emph{quaternum.net}. \url{https://www.quaternum.net}

\leavevmode\vadjust pre{\hypertarget{ref-mellet_blankblue_2020}{}}%
Mellet, M. (2020). Blank.Blue {[}Carnet{]}. \emph{Blank.blue}.
\url{https://blank.blue/}

\leavevmode\vadjust pre{\hypertarget{ref-perret_site_2020}{}}%
Perret, A. (2020). {Site d'Arthur Perret} {[}\{Carnet\}{]}.
\emph{arthurperret.fr}. \url{https://www.arthurperret.fr}

\leavevmode\vadjust pre{\hypertarget{ref-savelli_fenetres_2007}{}}%
Savelli, A. (2007). Fenêtres {Open Space} {[}Carnet{]}. \emph{Fenêtres
Open Space}. \url{https://annesavelli.fr/}

\leavevmode\vadjust pre{\hypertarget{ref-vitali-rosati_culture_2018}{}}%
Vitali-Rosati, M. (2018). Culture Numérique. {Pour} Une Philosophie Du
Numérique {[}Blogue{]}. \emph{Culture numérique.}
\url{http://blog.sens-public.org/marcellovitalirosati/}

\leavevmode\vadjust pre{\hypertarget{ref-walsh_melanie_2016}{}}%
Walsh, M. (2016). Melanie {Walsh} {[}Site{]}. \emph{Melanie Walsh}.
\url{https://melaniewalsh.org/melaniewalsh.org/}

\end{CSLReferences}}



\end{document}
