\documentclass[12pt,french,letterpaper]{article}
%déclaration du document 

\usepackage[T1]{fontenc}
%package pour police
\usepackage[utf8]{inputenc}
%package pour police

\usepackage[document]{ragged2e}
\usepackage{graphicx}
\usepackage{fancyhdr}
\usepackage{biblatex}
\usepackage{csquotes}

\usepackage{mathptmx}%choix de la police mathptmx = Times

\usepackage[dvipsnames,svgnames]{xcolor}%chois pour les couleurs


\usepackage{float}
\let\origfigure=\figure
\let\endorigfigure=\endfigure
\renewenvironment{figure}[1][]{%
  \origfigure[H]
}{%
  \endorigfigure
}

\usepackage{epigraph}
\usepackage{fancyhdr}


\providecommand{\subtitle}[1]{}
\subtitle{Hymne à l'enfant du conjoint}
\author{Alaska    Malo    Université de Montréal }
\providecommand{\cours}[1]{}
\cours{FRA3825 Pratiques de l'édition numérique}
\cours{}
\date{}


\pagestyle{fancy}
\fancyhf{}
\lhead{}
\rhead{}
\cfoot{Page \thepage}

\usepackage{graphicx}%package pour la gestion des images
\graphicspath{ {./media/} }%chemin des figures

\usepackage{graphicx,grffile}
\makeatletter
\def\maxwidth{\ifdim\Gin@nat@width>\linewidth\linewidth\else\Gin@nat@width\fi}
\def\maxheight{\ifdim\Gin@nat@height>\textheight\textheight\else\Gin@nat@height\fi}
\makeatother
% Scale images if necessary, so that they will not overflow the page
% margins by default, and it is still possible to overwrite the defaults
% using explicit options in \includegraphics[width, height, ...]{}
\setkeys{Gin}{width=\maxwidth,height=\maxheight,keepaspectratio}

\usepackage{amssymb,amsmath}
\usepackage{ifxetex,ifluatex}
\usepackage{fixltx2e} % provides \textsubscript
\ifnum 0\ifxetex 1\fi\ifluatex 1\fi=0 % if pdftex
  \usepackage[T1]{fontenc}


\else % if luatex or xelatex

  \usepackage{unicode-math}

  \defaultfontfeatures{Ligatures=TeX,Scale=MatchLowercase}
\fi
% use upquote if available, for straight quotes in verbatim environments
\IfFileExists{upquote.sty}{\usepackage{upquote}}{}
% use microtype if available
\IfFileExists{microtype.sty}{%
\usepackage[]{microtype}
\UseMicrotypeSet[protrusion]{basicmath} % disable protrusion for tt fonts
}{}
\PassOptionsToPackage{hyphens}{url} % url is loaded by hyperref

\usepackage[unicode=true]{hyperref}
\hypersetup{
            pdftitle={L'Anti-mère : Hymne à l'enfant du conjoint},
            pdfauthor={Alaska ,  Malo,  },
            colorlinks=true,
            urlcolor= RoyalBlue, 
	          linkcolor= Blue,
            pdfborder={0 0 0},
            breaklinks=true}
\urlstyle{same}  % don't use monospace font for urls



\IfFileExists{parskip.sty}{%
\usepackage{parskip}
}{% else
\setlength{\parindent}{0pt}
\setlength{\parskip}{6pt plus 2pt minus 1pt}
}
\setlength{\emergencystretch}{3em}  % prevent overfull lines
\providecommand{\tightlist}{%
  \setlength{\itemsep}{0pt}\setlength{\parskip}{0pt}}
\setcounter{secnumdepth}{0}
% Redefines (sub)paragraphs to behave more like sections
\ifx\paragraph\undefined\else
\let\oldparagraph\paragraph
\renewcommand{\paragraph}[1]{\oldparagraph{#1}\mbox{}}
\fi
\ifx\subparagraph\undefined\else
\let\oldsubparagraph\subparagraph
\renewcommand{\subparagraph}[1]{\oldsubparagraph{#1}\mbox{}}
\fi

% set default figure placement to htbp
\makeatletter
\def\fps@figure{htbp}
\makeatother




\begin{document}%début de mon document

\begin{titlepage}%début de ma page de titre
\begin{center}
    \enlargethispage{2cm}
    
\includegraphics[width = 50mm]{logo} %ajout de l'image en taille logo
%si le logo ne fonctionne pas : 
%\scshape{Université de Montréal} %utilisation des small caps 

\vspace*{3cm}
\scshape\Huge L'\emph{Anti-mère}\\
\normalfont\Large Hymne à l'enfant du conjoint\\
\large \vspace*{3cm}
Alaska,  Malo,  
\\
\normalsize\vspace*{1cm}Travail présenté dans le cadre du cours FRA3825 - \em Pratiques
de l'édition numérique
 \normalfont donné par Margot Mellet 

\vspace*{3cm}
\end{center}

\vspace*{\fill}
\begin{flushright}
\end{flushright}

\begin{center}
\scshape\normalsize\vspace*{1cm} 21-Mars-2023 --      Université de
Montréal 
\\
\end{center}
\end{titlepage}




\newpage 

\normalsize{\hypertarget{description-du-thuxe8me}{%
\subsection{Description du thème}\label{description-du-thuxe8me}}

Je parlerai d'abord de mon ancienne belle-mère~--~qui a l'étrange
particularité d'avoir le même prénom que ma mère~-- et de ses enfants.
Nous n'avons jamais eu une très bonne relation par leur amour pour le
sport et leurs tendances égocentriques ainsi que mes goûts artistiques
et ce qui est possiblement mon autisme. Je mettrais en lumière certains
souvenirs, mes perceptions sur les événements qui ont lieu pendant les
17 ans de couple de mon père et de cette \emph{Anti-mère}.

Ensuite, si j'ai le temps, je prendrais alors ma vraie mère, qui est
rendue à prendre le rôle de l'Anti-mère par son jugement dit «~terf~»,
par son refus de respecter mon nom et mon identité de genre.
L'\emph{Anti-mère} est à la fois cette figure maternelle qui demande le
respect mais ne le redonne pas en retour. C'est ce qu'est devenue ma
mère pendant les dernières années, c'est ce qu'était cette ancienne
belle-mère.

\hypertarget{techniques-et-uxe9dition}{%
\subsection{Techniques et édition}\label{techniques-et-uxe9dition}}

Je compte aussi faire ce travail sur le modèle de la Fabrique, pour
donner une allure de recueil de nouvelles. Aussi, cela avance très bien
avec ma compétence~--~ou plutôt mon incompétence~--~informatique et
m'éviterait de causer plus de trouble à mon ordinateur.

Je compte aussi utiliser expérimenter avec l'\emph{italique} pour
certains termes, dont «~\emph{Anti-mère}~» ou la dérivation de prénoms
de ceux qui ont inspirés ce futur projet. Pour donner un autre exemple,
le nom de cette ancienne belle-mère sera donc remplacé par
«~\emph{Einna}~», ou le nom de sa fille sera «~\emph{Esor}~». C'est une
méthode d'écriture que j'ai longtemps adopté et par cette ancienneté, je
ne compte pas m'en débarrasser de sitôt.

\hypertarget{inspirations-culturelles}{%
\subsection{Inspirations culturelles}\label{inspirations-culturelles}}

J'aimerais m'inspirer de certain.es chanteur.euses avec certaines
images. Je risque de faire allusion au groupe Against The Current, dont
leurs chansons m'ont grandement aidé avec mes émotions pendant mon
adolescence, plus précisément depuis mes 14 ans. Je compte citer
quelques vers avant d'introduire mes textes, comme pour ajouter ce qui
fut mon antitode.

Je compte aussi éditer des textes sur les belles-mères écrits par des
auteurs morts depuis plus de 70 ans. Par exemple, pour être dans
l'évidence, j'empruntrerais les contes «~Cendrillon~» à Charles Perrault
et «~Blanche-Neige~» aux Frères Grimm, ou encore des extraits de
\emph{Phèdre} de Jean Racine. Dans ces extraits, je ne laisserai pas une
parole de chanson, pour ne pas \emph{polluer} le texte d'origine.}



\end{document}
