\documentclass[12pt,french,letterpaper]{article}
%déclaration du document 

\usepackage[T1]{fontenc}
%package pour police
\usepackage[utf8]{inputenc}
%package pour police

\usepackage[document]{ragged2e}
\usepackage{graphicx}
\usepackage{fancyhdr}
\usepackage{biblatex}
\usepackage{csquotes}

\usepackage{mathptmx}%choix de la police mathptmx = Times

\usepackage[dvipsnames,svgnames]{xcolor}%chois pour les couleurs


\usepackage{float}
\let\origfigure=\figure
\let\endorigfigure=\endfigure
\renewenvironment{figure}[1][]{%
  \origfigure[H]
}{%
  \endorigfigure
}

\usepackage{epigraph}
\usepackage{fancyhdr}


\providecommand{\subtitle}[1]{}
\subtitle{sous-titre}
\author{Gabrielle    Lahaie    Université de Montréal }
\providecommand{\cours}[1]{}
\cours{FRA3825 Pratiques de l'édition numérique}
\cours{}
\date{}


\pagestyle{fancy}
\fancyhf{}
\lhead{}
\rhead{}
\cfoot{Page \thepage}

\usepackage{graphicx}%package pour la gestion des images
\graphicspath{ {./media/} }%chemin des figures

\usepackage{graphicx,grffile}
\makeatletter
\def\maxwidth{\ifdim\Gin@nat@width>\linewidth\linewidth\else\Gin@nat@width\fi}
\def\maxheight{\ifdim\Gin@nat@height>\textheight\textheight\else\Gin@nat@height\fi}
\makeatother
% Scale images if necessary, so that they will not overflow the page
% margins by default, and it is still possible to overwrite the defaults
% using explicit options in \includegraphics[width, height, ...]{}
\setkeys{Gin}{width=\maxwidth,height=\maxheight,keepaspectratio}

\usepackage{amssymb,amsmath}
\usepackage{ifxetex,ifluatex}
\usepackage{fixltx2e} % provides \textsubscript
\ifnum 0\ifxetex 1\fi\ifluatex 1\fi=0 % if pdftex
  \usepackage[T1]{fontenc}


\else % if luatex or xelatex

  \usepackage{unicode-math}

  \defaultfontfeatures{Ligatures=TeX,Scale=MatchLowercase}
\fi
% use upquote if available, for straight quotes in verbatim environments
\IfFileExists{upquote.sty}{\usepackage{upquote}}{}
% use microtype if available
\IfFileExists{microtype.sty}{%
\usepackage[]{microtype}
\UseMicrotypeSet[protrusion]{basicmath} % disable protrusion for tt fonts
}{}
\PassOptionsToPackage{hyphens}{url} % url is loaded by hyperref

\usepackage[unicode=true]{hyperref}
\hypersetup{
            pdftitle={Mon Portfolio : sous-titre},
            pdfauthor={Gabrielle ,  Lahaie,  },
            colorlinks=true,
            urlcolor= RoyalBlue, 
	          linkcolor= Blue,
            pdfborder={0 0 0},
            breaklinks=true}
\urlstyle{same}  % don't use monospace font for urls



\IfFileExists{parskip.sty}{%
\usepackage{parskip}
}{% else
\setlength{\parindent}{0pt}
\setlength{\parskip}{6pt plus 2pt minus 1pt}
}
\setlength{\emergencystretch}{3em}  % prevent overfull lines
\providecommand{\tightlist}{%
  \setlength{\itemsep}{0pt}\setlength{\parskip}{0pt}}
\setcounter{secnumdepth}{0}
% Redefines (sub)paragraphs to behave more like sections
\ifx\paragraph\undefined\else
\let\oldparagraph\paragraph
\renewcommand{\paragraph}[1]{\oldparagraph{#1}\mbox{}}
\fi
\ifx\subparagraph\undefined\else
\let\oldsubparagraph\subparagraph
\renewcommand{\subparagraph}[1]{\oldsubparagraph{#1}\mbox{}}
\fi

% set default figure placement to htbp
\makeatletter
\def\fps@figure{htbp}
\makeatother




\begin{document}%début de mon document

\begin{titlepage}%début de ma page de titre
\begin{center}
    \enlargethispage{2cm}
    
\includegraphics[width = 50mm]{logo} %ajout de l'image en taille logo
%si le logo ne fonctionne pas : 
%\scshape{Université de Montréal} %utilisation des small caps 

\vspace*{3cm}
\scshape\Huge Mon Portfolio\\
\normalfont\Large sous-titre\\
\large \vspace*{3cm}
Gabrielle,  Lahaie,  
\\
\normalsize\vspace*{1cm}Travail présenté dans le cadre du cours FRA3825 - \em Pratiques
de l'édition numérique
 \normalfont donné par Margot Mellet 

\vspace*{3cm}
\end{center}

\vspace*{\fill}
\begin{flushright}
\end{flushright}

\begin{center}
\scshape\normalsize\vspace*{1cm} 21-Mars-2023 --      Université de
Montréal 
\\
\end{center}
\end{titlepage}




\newpage 

\normalsize{proposition de projet d'écriture :

\begin{itemize}
\tightlist
\item
  Besoin d'écriture
\item
  Inspiration
\item
  Objectifs (technique, éditorial ou personnel)
\end{itemize}

Mes intentions pour mon carnet d'écriture sont personnelles mais dans le
but de m'exercer à l'édition et les techniques qui l'accompagne. Dans le
cadre d'un cours d'écriture de scénario nous avons interviewé une
écrivaine de jeux vidéo qui nous a donné quelques conseils de création
de portfolio. Elle a insisté sur le fait qu'un site web ainsi qu'un PDF
est important. Puisque j'aspire au monde des jeux vidéo, la compétition
est grande et avoir du bon matériel professionnel sous la main en tout
temps est primordial si on veut avoir la chance d'entrer dans
l'industrie. À la lumière de ces faits, je souhaite créer un carnet
d'écriture qui servira de portfolio professionnel de mes écrits pour
m'aider à trouver un travail après l'université. Ce portfolio comportera
mes meilleurs extraits d'écrits que je vais éditer. J'aimerais ainsi
m'entraîner à retravailler des textes considérés ``brouillons'' pour les
rendre propres et présentables en les éditant comme s'ils seraient
publiés. Les textes auront la possibilité d'être exportés en PDF pour
moi-même ou les potentiels recruteurs qui visiteront mon portfolio. Je
souhaite depuis longtemps créer mon propre portfolio pour plusieurs
domaines, donc j'aimerais construire des sections pour différents
métiers, voici ce que je pense en ce moment : une section de texte plus
en proses, une de scénarios de cinéma/télévision et une pour les
scénarios de jeux vidéo qui devra être en anglais. Avoir ce portfolio
sera pour moi une sécurité pour me trouver un emploi mais aussi une
manière de faire un portrait de moi-même pour voir le niveau où je suis
présentement. Je veux le garder à jours quand je ferai de meilleurs
textes, mais conserver des traces de mon progrès dans une section
cachée.}



\end{document}
